\begin{abstract}
Multi-sensor platforms like buoys and gliders produce one or more readings per sensor on varying, discrete time frequencies.  The resulting dataset is a sparse matrix with rows containing readings from sensors that reported at a moment in time and �NaN� for missing readings from sensors that did not report at that moment.  To assist with data analysis, datasets are inserted into relational database schemas.  These schemas must either be normalized with replicate data (e.g., time, latitude, longitude, and depth) across tables to minimize the number of joins necessary for a data query or be built with large tables containing NULL values in every row.  NoSQL databases provide a schema-less alternative for storing and retrieving sparse datasets.

Relational databases are appropriate for circumstances where the data structure is well-defined and mostly static.  Business processes, for example, can be well-defined and remain static through the life of the company.  Multi-sensor platforms, on the other hand, are dynamic.  Sensors may be added or removed and field transmissions may only return partial data.  These inconsistencies lead to complex schemas and inefficient relational databases.

As datasets have grown larger and more dynamic in web-based applications, NoSQL database types have grown in popularity.  NoSQL databases, in general, are built across multiple servers.  Database clustering and replicas make these systems scalable and robust.   Distributing data is possible because most NoSQL systems have no overarching schema.  Instead, the system stores data points in records that each have a unique identifier (key) and their own data structure (value) with one or more fields.  Records are organized into collections which are logically together in the database client libraries, but may contain records that are physically distributed.  Queries are performed by specifying the collection of records to query and logical parameters for record fields.  Across server indexes on record fields can be created to speed up queries.

Glider Database Alternative with Mongo (GDAM) is a data management system for gliders built on the MongoDB NoSQL database engine.  It is live in production at the University of South Florida Center for Ocean Technology (USF COT).  GDAM is a collection of scripts which parse, process and store real-time glider datasets.  Data is parsed as soon as it is transmitted via satellite to our shore-based servers.  A collection exists for each glider.  Each record in a collection is a moment in time on the glider when a subset of available sensors reported.  The system has been tested during two Slocum G1 glider deployments in September and October of 2012.  Archival datasets dating back to March of 2009 have also been uploaded into this system.  As of May 2013, there are 685,977 records available for three different gliders and 22 deployments.  Records are indexed by time, GPS, and depth with the ability to add more indexes as necessary.

The paper compares GDAM with the traditional relational database schemas used for glider operations.  It begins by reviewing current database systems in use for deployable glider systems.  Next, the implementation of the GDAM system is discussed in detail.  After discussing both current systems and a NoSQL implementation, the paper compares the two in terms of capabilities, physical storage requirements, performance, and scalability.  The remainder of the paper discusses directions for future work and draws a conclusion based on the comparison of relational and NoSQL systems.
\end{abstract}